\documentclass[]{tufte-handout}

% ams
\usepackage{amssymb,amsmath}

\usepackage{ifxetex,ifluatex}
\usepackage{fixltx2e} % provides \textsubscript
\ifnum 0\ifxetex 1\fi\ifluatex 1\fi=0 % if pdftex
  \usepackage[T1]{fontenc}
  \usepackage[utf8]{inputenc}
\else % if luatex or xelatex
  \makeatletter
  \@ifpackageloaded{fontspec}{}{\usepackage{fontspec}}
  \makeatother
  \defaultfontfeatures{Ligatures=TeX,Scale=MatchLowercase}
  \makeatletter
  \@ifpackageloaded{soul}{
     \renewcommand\allcapsspacing[1]{{\addfontfeature{LetterSpace=15}#1}}
     \renewcommand\smallcapsspacing[1]{{\addfontfeature{LetterSpace=10}#1}}
   }{}
  \makeatother

\fi

% graphix
\usepackage{graphicx}
\setkeys{Gin}{width=\linewidth,totalheight=\textheight,keepaspectratio}

% booktabs
\usepackage{booktabs}

% url
\usepackage{url}

% hyperref
\usepackage{hyperref}

% units.
\usepackage{units}


\setcounter{secnumdepth}{-1}

% citations

% pandoc syntax highlighting
\usepackage{color}
\usepackage{fancyvrb}
\newcommand{\VerbBar}{|}
\newcommand{\VERB}{\Verb[commandchars=\\\{\}]}
\DefineVerbatimEnvironment{Highlighting}{Verbatim}{commandchars=\\\{\}}
% Add ',fontsize=\small' for more characters per line
\newenvironment{Shaded}{}{}
\newcommand{\AlertTok}[1]{\textcolor[rgb]{1.00,0.00,0.00}{\textbf{#1}}}
\newcommand{\AnnotationTok}[1]{\textcolor[rgb]{0.38,0.63,0.69}{\textbf{\textit{#1}}}}
\newcommand{\AttributeTok}[1]{\textcolor[rgb]{0.49,0.56,0.16}{#1}}
\newcommand{\BaseNTok}[1]{\textcolor[rgb]{0.25,0.63,0.44}{#1}}
\newcommand{\BuiltInTok}[1]{#1}
\newcommand{\CharTok}[1]{\textcolor[rgb]{0.25,0.44,0.63}{#1}}
\newcommand{\CommentTok}[1]{\textcolor[rgb]{0.38,0.63,0.69}{\textit{#1}}}
\newcommand{\CommentVarTok}[1]{\textcolor[rgb]{0.38,0.63,0.69}{\textbf{\textit{#1}}}}
\newcommand{\ConstantTok}[1]{\textcolor[rgb]{0.53,0.00,0.00}{#1}}
\newcommand{\ControlFlowTok}[1]{\textcolor[rgb]{0.00,0.44,0.13}{\textbf{#1}}}
\newcommand{\DataTypeTok}[1]{\textcolor[rgb]{0.56,0.13,0.00}{#1}}
\newcommand{\DecValTok}[1]{\textcolor[rgb]{0.25,0.63,0.44}{#1}}
\newcommand{\DocumentationTok}[1]{\textcolor[rgb]{0.73,0.13,0.13}{\textit{#1}}}
\newcommand{\ErrorTok}[1]{\textcolor[rgb]{1.00,0.00,0.00}{\textbf{#1}}}
\newcommand{\ExtensionTok}[1]{#1}
\newcommand{\FloatTok}[1]{\textcolor[rgb]{0.25,0.63,0.44}{#1}}
\newcommand{\FunctionTok}[1]{\textcolor[rgb]{0.02,0.16,0.49}{#1}}
\newcommand{\ImportTok}[1]{#1}
\newcommand{\InformationTok}[1]{\textcolor[rgb]{0.38,0.63,0.69}{\textbf{\textit{#1}}}}
\newcommand{\KeywordTok}[1]{\textcolor[rgb]{0.00,0.44,0.13}{\textbf{#1}}}
\newcommand{\NormalTok}[1]{#1}
\newcommand{\OperatorTok}[1]{\textcolor[rgb]{0.40,0.40,0.40}{#1}}
\newcommand{\OtherTok}[1]{\textcolor[rgb]{0.00,0.44,0.13}{#1}}
\newcommand{\PreprocessorTok}[1]{\textcolor[rgb]{0.74,0.48,0.00}{#1}}
\newcommand{\RegionMarkerTok}[1]{#1}
\newcommand{\SpecialCharTok}[1]{\textcolor[rgb]{0.25,0.44,0.63}{#1}}
\newcommand{\SpecialStringTok}[1]{\textcolor[rgb]{0.73,0.40,0.53}{#1}}
\newcommand{\StringTok}[1]{\textcolor[rgb]{0.25,0.44,0.63}{#1}}
\newcommand{\VariableTok}[1]{\textcolor[rgb]{0.10,0.09,0.49}{#1}}
\newcommand{\VerbatimStringTok}[1]{\textcolor[rgb]{0.25,0.44,0.63}{#1}}
\newcommand{\WarningTok}[1]{\textcolor[rgb]{0.38,0.63,0.69}{\textbf{\textit{#1}}}}

% longtable

% multiplecol
\usepackage{multicol}

% strikeout
\usepackage[normalem]{ulem}

% morefloats
\usepackage{morefloats}


% tightlist macro required by pandoc >= 1.14
\providecommand{\tightlist}{%
  \setlength{\itemsep}{0pt}\setlength{\parskip}{0pt}}

% title / author / date
\title{introduction\_to\_R}
\author{Will Mackey}
\date{19/02/2019}


\begin{document}

\maketitle




\hypertarget{preamble}{%
\section{Preamble}\label{preamble}}

\hypertarget{why-we-are-doing-this}{%
\subsection{Why we are doing this}\label{why-we-are-doing-this}}

\hypertarget{a-script-based-language}{%
\subsubsection{A script-based language}\label{a-script-based-language}}

R is a script-based language. You write down a list of instructions and
it will follow, performing one action after another. This is different
to `point and click' software like Microsoft Excel, and it can feel a
bit cumbersome.

In Excel, you can perform a series of steps:

\begin{itemize}
\tightlist
\item
  Open a file.
\item
  Delete column that you don't need.
\item
  Add a new column that contains a function.
\item
  `Fill' that function down to the end of the dataset.
\item
  Select all of your data and sort from highest to lowest.
\item
  Delete all the rows that have missing values.
\item
  Save your file.
\end{itemize}

R is free, open-source and powerful. In the past five years, it has also
become easier to use and get started with.

\ldots{}

\hypertarget{getting-started-in-r}{%
\subsection{Getting started in R}\label{getting-started-in-r}}

\hypertarget{r-projects}{%
\subsubsection{\texorpdfstring{R
\emph{Projects}}{R Projects}}\label{r-projects}}

\hypertarget{functions}{%
\subsubsection{Functions}\label{functions}}

\hypertarget{installing-and-loading-packages}{%
\subsubsection{Installing and loading
packages}\label{installing-and-loading-packages}}

Installing a package is like installing an app on your phone.
It\ldots{}.

You can install a package using the \texttt{install.packages} function.

\begin{Shaded}
\begin{Highlighting}[]
\KeywordTok{install.packages}\NormalTok{(}\StringTok{"tidyverse"}\NormalTok{)}
\end{Highlighting}
\end{Shaded}

Now we need to load it using the \texttt{library} function; like opening
an app you have installed on your phone. We do this every time (every
`session') we want to use it.

\begin{Shaded}
\begin{Highlighting}[]
\KeywordTok{library}\NormalTok{(tidyverse)}
\end{Highlighting}
\end{Shaded}

\begin{verbatim}
## -- Attaching packages --------------------------------------------------------------------------------------------------------------------------------------------------- tidyverse 1.2.1 --
\end{verbatim}

\begin{verbatim}
## v ggplot2 3.1.0     v purrr   0.2.5
## v tibble  2.0.1     v dplyr   0.7.8
## v tidyr   0.8.2     v stringr 1.4.0
## v readr   1.3.1     v forcats 0.3.0
\end{verbatim}

\begin{verbatim}
## Warning: package 'tibble' was built under R
## version 3.5.2
\end{verbatim}

\begin{verbatim}
## Warning: package 'stringr' was built under R
## version 3.5.2
\end{verbatim}

\begin{verbatim}
## -- Conflicts ------------------------------------------------------------------------------------------------------------------------------------------------------ tidyverse_conflicts() --
## x dplyr::filter() masks stats::filter()
## x dplyr::lag()    masks stats::lag()
\end{verbatim}

\hypertarget{using-r-part-1}{%
\section{Using R: Part 1}\label{using-r-part-1}}

This

\hypertarget{read-a-csv-file-into-r}{%
\subsection{Read a CSV file into R}\label{read-a-csv-file-into-r}}

This uses the \texttt{read\_csv} function and, here, we're only going to
give it one argument: \footnote{footnote content} \# with one argument:
the path to the csv file you want to read. \# Tip: open quotation marks
and hit `tab' to choose your file. \# (and save you some typing)

\begin{Shaded}
\begin{Highlighting}[]
\KeywordTok{read_csv}\NormalTok{(}\StringTok{"data/gapminder.csv"}\NormalTok{)}
\end{Highlighting}
\end{Shaded}

\begin{verbatim}
## Parsed with column specification:
## cols(
##   country = col_character(),
##   continent = col_character(),
##   year = col_double(),
##   lifeExp = col_double(),
##   pop = col_double(),
##   gdpPercap = col_double()
## )
\end{verbatim}

\begin{verbatim}
## # A tibble: 1,704 x 6
##    country continent  year lifeExp    pop
##    <chr>   <chr>     <dbl>   <dbl>  <dbl>
##  1 Afghan~ Asia       1952    28.8 8.43e6
##  2 Afghan~ Asia       1957    30.3 9.24e6
##  3 Afghan~ Asia       1962    32.0 1.03e7
##  4 Afghan~ Asia       1967    34.0 1.15e7
##  5 Afghan~ Asia       1972    36.1 1.31e7
##  6 Afghan~ Asia       1977    38.4 1.49e7
##  7 Afghan~ Asia       1982    39.9 1.29e7
##  8 Afghan~ Asia       1987    40.8 1.39e7
##  9 Afghan~ Asia       1992    41.7 1.63e7
## 10 Afghan~ Asia       1997    41.8 2.22e7
## # ... with 1,694 more rows, and 1 more
## #   variable: gdpPercap <dbl>
\end{verbatim}

Looks good! But it isn't in our Environment (on the right) yet because
we didn't \emph{assign it} to anything.

\begin{Shaded}
\begin{Highlighting}[]
\NormalTok{gapminder <-}\StringTok{ }\KeywordTok{read_csv}\NormalTok{(}\StringTok{"data/gapminder.csv"}\NormalTok{)  }\CommentTok{# the 'assign' operator is <-}
\end{Highlighting}
\end{Shaded}

\begin{verbatim}
## Parsed with column specification:
## cols(
##   country = col_character(),
##   continent = col_character(),
##   year = col_double(),
##   lifeExp = col_double(),
##   pop = col_double(),
##   gdpPercap = col_double()
## )
\end{verbatim}

\begin{Shaded}
\begin{Highlighting}[]
\CommentTok{# ie assign read_csv(...) to gapminder}
\end{Highlighting}
\end{Shaded}

Now it is in our Global Environment over there ---\textgreater{} wooh!

\hypertarget{peeking-at-the-data}{%
\subsubsection{Peeking at the data}\label{peeking-at-the-data}}

Much like Excel, we can explore the gapminder dataset with our eyes.

\begin{Shaded}
\begin{Highlighting}[]
\KeywordTok{head}\NormalTok{(gapminder)  }\CommentTok{# see the 'head' of the data}
\end{Highlighting}
\end{Shaded}

\begin{verbatim}
## # A tibble: 6 x 6
##   country continent  year lifeExp    pop
##   <chr>   <chr>     <dbl>   <dbl>  <dbl>
## 1 Afghan~ Asia       1952    28.8 8.43e6
## 2 Afghan~ Asia       1957    30.3 9.24e6
## 3 Afghan~ Asia       1962    32.0 1.03e7
## 4 Afghan~ Asia       1967    34.0 1.15e7
## 5 Afghan~ Asia       1972    36.1 1.31e7
## 6 Afghan~ Asia       1977    38.4 1.49e7
## # ... with 1 more variable: gdpPercap <dbl>
\end{verbatim}

\begin{Shaded}
\begin{Highlighting}[]
\CommentTok{# View(gapminder) # View the whole dataset in}
\CommentTok{# a separate window Hey it looks just like an}
\CommentTok{# Excel spreadsheet!}
\end{Highlighting}
\end{Shaded}

\hypertarget{visualising-the-data}{%
\subsection{Visualising the data}\label{visualising-the-data}}

\begin{Shaded}
\begin{Highlighting}[]
\CommentTok{## Scatter plot}
\NormalTok{gapminder }\OperatorTok\StringTok{ }\KeywordTok{ggplot}\NormalTok{(}\KeywordTok{aes}\NormalTok{(}\DataTypeTok{x =}\NormalTok{ lifeExp, }\DataTypeTok{y =}\NormalTok{ gdpPercap)) }\OperatorTok{+}\StringTok{ }
\StringTok{    }\KeywordTok{geom_point}\NormalTok{()}
\end{Highlighting}
\end{Shaded}

\includegraphics{introduction_to_R_files/figure-latex/unnamed-chunk-6-1}

\hypertarget{look-at-the-gapminder-dataset.}{%
\subsection{Look at the gapminder
dataset.}\label{look-at-the-gapminder-dataset.}}

Now close your eyes and picture the gapminder dataset: * Add a new
column to the right with the name `my\_column'. * Only keep rows from
2007 * Then remove the `year' column

\hypertarget{putting-it-all-together-with-pipes}{%
\section{Putting it all together with pipes
\%\textgreater{}\%}\label{putting-it-all-together-with-pipes}}

\begin{Shaded}
\begin{Highlighting}[]
\NormalTok{gapminder07 <-}\StringTok{ }\NormalTok{gapminder }\OperatorTok\StringTok{         }\CommentTok{# Assign gapminder07 to: the gapminder dataset, then}
\StringTok{  }
\StringTok{  }\KeywordTok{mutate}\NormalTok{(}\DataTypeTok{gdp =}\NormalTok{ gdpPercap }\OperatorTok{*}\StringTok{ }\NormalTok{pop) }\OperatorTok\StringTok{  }\CommentTok{# create a new column called gdp, then}
\StringTok{  }
\StringTok{  }\KeywordTok{filter}\NormalTok{(year }\OperatorTok{==}\StringTok{ }\DecValTok{2007}\NormalTok{) }\OperatorTok\StringTok{           }\CommentTok{# keep only observations from 2007, then}
\StringTok{  }
\StringTok{  }\KeywordTok{select}\NormalTok{(}\OperatorTok{-}\NormalTok{gdpPercap)                 }\CommentTok{# drop the gdpPercap variable (negative select)}
\end{Highlighting}
\end{Shaded}

\hypertarget{changing-the-look-of-our-plots}{%
\subsection{Changing the look of our
plots}\label{changing-the-look-of-our-plots}}

We want to make our plots as clear as possible\ldots{}

\begin{Shaded}
\begin{Highlighting}[]
\KeywordTok{library}\NormalTok{(scales)}
\end{Highlighting}
\end{Shaded}

\begin{verbatim}
## 
## Attaching package: 'scales'
\end{verbatim}

\begin{verbatim}
## The following object is masked from 'package:purrr':
## 
##     discard
\end{verbatim}

\begin{verbatim}
## The following object is masked from 'package:readr':
## 
##     col_factor
\end{verbatim}

\begin{Shaded}
\begin{Highlighting}[]
\CommentTok{# with a log scale}
\NormalTok{gapminder07 }\OperatorTok\StringTok{ }\KeywordTok{ggplot}\NormalTok{(}\KeywordTok{aes}\NormalTok{(}\DataTypeTok{x =}\NormalTok{ lifeExp, }\DataTypeTok{y =}\NormalTok{ gdp)) }\OperatorTok{+}\StringTok{ }
\StringTok{    }\KeywordTok{geom_point}\NormalTok{() }\OperatorTok{+}\StringTok{ }\KeywordTok{scale_y_log10}\NormalTok{(}\DataTypeTok{label =}\NormalTok{ comma)}
\end{Highlighting}
\end{Shaded}

\includegraphics{introduction_to_R_files/figure-latex/unnamed-chunk-8-1}

\begin{Shaded}
\begin{Highlighting}[]
\CommentTok{# with colour}
\NormalTok{gapminder07 }\OperatorTok\StringTok{ }\KeywordTok{ggplot}\NormalTok{(}\KeywordTok{aes}\NormalTok{(}\DataTypeTok{x =}\NormalTok{ lifeExp, }\DataTypeTok{y =}\NormalTok{ gdp, }
    \DataTypeTok{colour =}\NormalTok{ continent)) }\OperatorTok{+}\StringTok{ }\KeywordTok{geom_point}\NormalTok{() }\OperatorTok{+}\StringTok{ }\KeywordTok{geom_line}\NormalTok{(}\KeywordTok{aes}\NormalTok{(}\DataTypeTok{group =}\NormalTok{ country)) }\OperatorTok{+}\StringTok{ }
\StringTok{    }\KeywordTok{scale_y_log10}\NormalTok{(}\DataTypeTok{label =}\NormalTok{ comma)}
\end{Highlighting}
\end{Shaded}

\begin{verbatim}
## geom_path: Each group consists of only one
## observation. Do you need to adjust the
## group aesthetic?
\end{verbatim}

\includegraphics{introduction_to_R_files/figure-latex/unnamed-chunk-8-2}

\begin{Shaded}
\begin{Highlighting}[]
\CommentTok{# with colour and facet}
\NormalTok{gapminder07 }\OperatorTok\StringTok{ }\KeywordTok{ggplot}\NormalTok{(}\KeywordTok{aes}\NormalTok{(}\DataTypeTok{x =}\NormalTok{ lifeExp, }\DataTypeTok{y =}\NormalTok{ gdp, }
    \DataTypeTok{colour =}\NormalTok{ continent)) }\OperatorTok{+}\StringTok{ }\KeywordTok{geom_point}\NormalTok{() }\OperatorTok{+}\StringTok{ }\KeywordTok{geom_line}\NormalTok{(}\KeywordTok{aes}\NormalTok{(}\DataTypeTok{group =}\NormalTok{ country)) }\OperatorTok{+}\StringTok{ }
\StringTok{    }\KeywordTok{scale_y_log10}\NormalTok{(}\DataTypeTok{label =}\NormalTok{ comma) }\OperatorTok{+}\StringTok{ }\KeywordTok{facet_wrap}\NormalTok{(}\OperatorTok{~}\NormalTok{continent)}
\end{Highlighting}
\end{Shaded}

\begin{verbatim}
## geom_path: Each group consists of only one
## observation. Do you need to adjust the
## group aesthetic?
\end{verbatim}

\begin{verbatim}
## geom_path: Each group consists of only one
## observation. Do you need to adjust the
## group aesthetic?
## geom_path: Each group consists of only one
## observation. Do you need to adjust the
## group aesthetic?
## geom_path: Each group consists of only one
## observation. Do you need to adjust the
## group aesthetic?
## geom_path: Each group consists of only one
## observation. Do you need to adjust the
## group aesthetic?
\end{verbatim}

\includegraphics{introduction_to_R_files/figure-latex/unnamed-chunk-8-3}

\begin{Shaded}
\begin{Highlighting}[]
\CommentTok{# with colour and facet}
\NormalTok{gapminder07 }\OperatorTok\StringTok{ }\KeywordTok{mutate}\NormalTok{(}\DataTypeTok{decade =} \KeywordTok{signif}\NormalTok{(year, }\DecValTok{3}\NormalTok{)) }\OperatorTok\StringTok{ }
\StringTok{    }\KeywordTok{ggplot}\NormalTok{(}\KeywordTok{aes}\NormalTok{(}\DataTypeTok{x =}\NormalTok{ lifeExp, }\DataTypeTok{y =}\NormalTok{ gdp, }\DataTypeTok{colour =}\NormalTok{ continent, }
        \DataTypeTok{size =}\NormalTok{ pop)) }\OperatorTok{+}\StringTok{ }\KeywordTok{geom_point}\NormalTok{() }\OperatorTok{+}\StringTok{ }\KeywordTok{geom_line}\NormalTok{(}\KeywordTok{aes}\NormalTok{(}\DataTypeTok{group =}\NormalTok{ country)) }\OperatorTok{+}\StringTok{ }
\StringTok{    }\KeywordTok{scale_y_log10}\NormalTok{(}\DataTypeTok{label =}\NormalTok{ comma) }\OperatorTok{+}\StringTok{ }\KeywordTok{facet_grid}\NormalTok{(decade }\OperatorTok{~}\StringTok{ }
\StringTok{    }\NormalTok{continent)}
\end{Highlighting}
\end{Shaded}

\begin{verbatim}
## geom_path: Each group consists of only one
## observation. Do you need to adjust the
## group aesthetic?
## geom_path: Each group consists of only one
## observation. Do you need to adjust the
## group aesthetic?
## geom_path: Each group consists of only one
## observation. Do you need to adjust the
## group aesthetic?
## geom_path: Each group consists of only one
## observation. Do you need to adjust the
## group aesthetic?
## geom_path: Each group consists of only one
## observation. Do you need to adjust the
## group aesthetic?
\end{verbatim}

\includegraphics{introduction_to_R_files/figure-latex/unnamed-chunk-8-4}

\begin{Shaded}
\begin{Highlighting}[]
\CommentTok{# with colour}
\NormalTok{gapminder07 }\OperatorTok\StringTok{ }\KeywordTok{ggplot}\NormalTok{(}\KeywordTok{aes}\NormalTok{(}\DataTypeTok{x =}\NormalTok{ lifeExp, }\DataTypeTok{y =}\NormalTok{ gdp, }
    \DataTypeTok{colour =}\NormalTok{ continent)) }\OperatorTok{+}\StringTok{ }\KeywordTok{geom_point}\NormalTok{(}\DataTypeTok{alpha =} \FloatTok{0.5}\NormalTok{) }\OperatorTok{+}\StringTok{ }
\StringTok{    }\KeywordTok{geom_line}\NormalTok{(}\KeywordTok{aes}\NormalTok{(}\DataTypeTok{group =}\NormalTok{ country)) }\OperatorTok{+}\StringTok{ }\KeywordTok{scale_y_log10}\NormalTok{() }\OperatorTok{+}\StringTok{ }
\StringTok{    }\KeywordTok{facet_wrap}\NormalTok{(}\OperatorTok{~}\NormalTok{continent)}
\end{Highlighting}
\end{Shaded}

\begin{verbatim}
## geom_path: Each group consists of only one
## observation. Do you need to adjust the
## group aesthetic?
## geom_path: Each group consists of only one
## observation. Do you need to adjust the
## group aesthetic?
## geom_path: Each group consists of only one
## observation. Do you need to adjust the
## group aesthetic?
## geom_path: Each group consists of only one
## observation. Do you need to adjust the
## group aesthetic?
## geom_path: Each group consists of only one
## observation. Do you need to adjust the
## group aesthetic?
\end{verbatim}

\includegraphics{introduction_to_R_files/figure-latex/unnamed-chunk-8-5}



\end{document}
